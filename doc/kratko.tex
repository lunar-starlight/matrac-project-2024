\documentclass{article}
\usepackage{unicode-math}
\usepackage{enumerate}
\usepackage[backend=biber]{biblatex}
\bibliography{kratko.bib}
\begin{document}
\section{Formalization of sheaf forcing}
Defining a sheaf topos in a proof assistant is a difficult task.
The main problem is, that the nature of a topology is very impredicative,
so it does not easily lend itself to formalization efforts.
This is why the original description of the project was rather vague.

As a consequence it also means it took me quite a substantial amount of time
to work out the details of what I can even start with, so there is no sustantial work
done in the repo.

Certain topological spaces sidestep the issues above entirely, however.
The Cantor space $2^ℕ$ is one such space, and so is the Baire space $ℕ^ℕ$.
The topologies of these are such that any open cover can be refined to a partition,
so the sheaf gluing axioms become trivial (there are no intersections to verify)
and thus computers can work with them.

Therefore the current goal is to work along with \cite{coquand,dialogue}
to show using sheaf forcing over these spaces
that a certain class of functions is continuous, and in particular this
yields an algorithm which computes the modulus of convergence for these funcitons.

\section{Goals}

I have the following checkpoints in mind, and they may also serve as good endpoints for the project,
depending on how much time it will take to complete:
\begin{enumerate}[(1)]
	\item\label{cat} Define the category of sheaves over $2^ℕ$
	\item\label{forcing} Define sheaf forcing
	\item\label{syst} Define/interpret Gödel's System T
	\item\label{cont} Show that System T definable $2^ℕ → ℕ$ functions are uniformly continuous
	\item\label{baire} Extend \ref{cat} and \ref{forcing} to $ℕ^ℕ$
	\item Show the proof of \ref{cont} doesn't apply to \ref{baire}
	\item Prove \ref{cont} for \ref{baire} in the non-uniform case
\end{enumerate}

I plan to use the libraries agda-stdlib\cite{stdlib}, agda-categories\cite{catlib}, and also TypeTopology\cite{tytop} if absolutely necessary.

In case there are significant issues to defining the category itself,
I will attempt to directly describe forcing without mentioning sheaves,
as is (I believe) done in \cite{coquand}.

The description of the objects of interest in the Cantor space case is sufficiently finite
(it is precisely finite binary trees), so that computers can easily compute with them,
however the Baire space the trees are not finitely branching
(there is a branch for each natural number) so I believe more care needs
to be taken to define them correctly.

\printbibliography
\end{document}
